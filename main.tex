\documentclass{smarthepnote}
\usepackage[colorinlistoftodos]{todonotes}
\usepackage{placeins}


\title{SMARTHEP Whitepaper Title}
\author{The SMARTHEP Network}
\date{\today}

% Here is the information that will be entered in the title page
\DocAuthors{A.~Person \\ B.~Person \\ C.~Person}
\DocEditors{ D.~Person \\ E.~Person}
\DocCoordinators{F.~Person \\ G.~Person }
\DeliverableNo{xxx-yyy-zzzz}
\draftversion{0.1}

\begin{document}
\maketitle

\begin{abstract}
For this style of document, the abstract has more the function of a summary. 
\end{abstract}

% Make the review table at the bottom of the title page
\vfill
\makereviewtable
\clearpage

% Short documentes dont always need a Table of Content / Figures / Tables, so comment out what is not needed
%\begingroup
%\color{black}
%\tableofcontents
%\listoffigures
%\listoftables
%\endgroup
%\pagebreak

\section{Introduction}
Its always good to have an introduction, if only to have an example for a section. And here is an example for a reference from the bibtex file (see \cite{einstein}). Its also pretty easy to reference figures (see Figure \ref{fig:examplecernlogo}). \\
\begin{figure}[ht]
\centering
\includegraphics[width=0.5\textwidth]{images/cernlogo.eps}
\caption{\label{fig:examplecernlogo} Example of how to include a figure. This works with all sorts of formats, eps, pdf, png.}
\end{figure}

% this will prevent float objects like figures to be moved past this point in the document.
\FloatBarrier


You also have the option of using colored text, for example \color{blue} this part in blue,  \color{red} this part in red  \color{green} and this part in green, before \color{black} going back to black.  

\begin{enumerate}
\item Everyone loves an enumerated list.
\item If you prefer bulleted lists, see below.
\end{enumerate}

Of course there are always use cases for list with enumerations, and lists with bullets only, which is why it is useful to have examples of both.

\begin{itemize}
    \item Everyone loves a bulleted list.
    \item If you prefer an enumerated list, see above.
\end{itemize}

\section{The first section after the introduction}
Since an example of a ToC is not much fun with only one section, lets make another one and throw in some subsections as well.


\subsection{The first sub-section}
How about structuring the document into more subsections.

\subsection{The second sub-section}
Tables are just as easy as figures to construct and reference, for example this one here (see Table \ref{tab:exampletable}).

\begin{table}[h]
\begin{center}
\begin{tabular}{ |c|c|c|p{0.1\textwidth}| } 
    \hline
    \rowcolor{lightgray} 
    col1 & col2 & col3 & col4\\
    \hline
    \multirow{3}{4em}{Multiple row} & cell2 & cell3 & cell4 \\ 
    \cline{3-4}
    & cell5 & \multicolumn{2}{c|}{cell6 and cell7} \\
    \cline{3-4}
    & cell8 & cell9 & cell10 \\ 
    \hline
    cell11 & cell12 & cell13 & cell14 \\ 
    \hline
    \multicolumn{4}{|c|}{ Multicolumn} \\
    \hline
\end{tabular}
\end{center}
\caption{A table is as happy about a caption as a fiugre.}
\label{tab:exampletable}
\end{table}

\section{Conclusions}
If the document purpose calls for conclusions, this would be the place to put them.

\section*{Acknowledgements}
%This is important, as it acknowledges our funding. 
This work is part of the SMARTHEP network and it is funded by the European Union’s Horizon 2020 research and innovation programme, call H2020-MSCA-ITN-2020, under Grant Agreement n. 956086. 

\bibliography{references}
\bibliographystyle{JHEP}

\end{document}
